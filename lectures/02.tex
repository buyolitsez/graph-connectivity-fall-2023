
\section{Структура двусвязного графа}

\subsection{Дерево разбиения двусвязного графа}

\todo[inline]{Дописать начало лекции 2}

\begin{df}[Дерево разбиения $\BT(G)$]
	\textbf{Дерево разбиения} $\BT(G)$ двусвязного графа  $G$ - это дерево  $T(G, \D(G))$, см Определение \ref{definition:tree_of_partition}.

	Будем обозначать $\Part(G) = \Part(\D(G))$.
	Частями графа  $G$ будем называть элементы  $\Part(G)$.
\end{df}

\begin{df}[Крайняя часть]
	Часть $A \in \Part(\D(G))$ назовем \textbf{крайней}, если она соответствует висячей вершине дерева разбиения  $\BT(G)$.
\end{df}

\begin{df}[G']
	Для двусвязного графа $G$ обозначим через $G'$ граф, полученный из $G$ добавлением всех отсутствующих в $E(G)$ ребер вида $ab$, где $\{a, b\} \in \D(G)$.

	Т.е. $G' = G^{\D(G)}$
\end{df}

\begin{df}[Цикл и 3-блок]
	Назовем часть $A \in \Part(G)$ \textbf{циклом}, если граф $G'(A)$ - простой цикл и \textbf{3-блоком}, если граф $G'(A)$ трёхсвязен.
	Если часть $A$ - цикл, то мы будем называть $|A|$ \textbf{длиной} цикла.
\end{df}

\begin{lm}[Лемма 2.2] \label{lemma:2_2}
	Пусть $S \in \R_2(G)$ - не одиночное множество, $S \subset A \in \Part(G)$. 
	Тогда $S \in \R_2(G'(A))$, причем это множество - не одиночное и в  $G'(A)$.
\end{lm}

\begin{proof}
	Пусть $S' \in \R_2(G)$ - зависимое с  $S$ множество.

	По Лемме \ref{lemma:1_1} получаем что $S, S' \in \R_2(G')$ и эти множества зависимы в графе $G'$.

	По Лемме \ref{lemma:1_2} $G'(A)$ двусвязный, а значит нельзя разделить две вершины множества  $S \subset A$ в графе  $G'$, удалив менее двух вершин из части  $A$.

	Следовательно,  $S' \subset A$(ведь он разделяет $S$ и при этом $|S'| = 2$).

	Тогда $S$ и  $S'$ разделяют друг друга и в графе $G'(A)$.

	Следовательно,  $S, S' \in \R_2(G'(A))$, причем эти множества зависимы.

\end{proof}

\begin{lm}[Лемма 2.3] \label{lemma:2_3}
	Пусть $S = \{ a, b \} \in \R_2(G)$ не одиночное множество.
	Тогда $|\Part(S)| = 2$ и для каждой части $A \in \Part(S)$ граф $G(A)$ имеет точку сочленения, отделяющую $a$ от $b$.
\end{lm}

\begin{proof}
	Т.к. $S$ не одиночное, то возьмем зависимое с ним $S' \in \R_2(G)$.

	Множество $S'$ разделяет $S$, а значит не существует $ab$ пути по вершинам части $A$ в графе $G$ который не пересекается с $S'$(потому что тогда бы $S'$ не отделяло $a$ от $b$).
	Но если $S'$ не пересекает $\Int(A)$, то такой путь есть(ведь $A$ связно).

	А значит всего частей в $\Part(S)$ ровно две(ведь их внутренности не пересекаются).

	Более того, если $\{x\} = S' \cap \Int(A)$, то $x$ — отделяет $a$ от $b$ в $G(A)$ по сказанному выше.
	
\end{proof}

\begin{thm}[Теорема 2.1] \label{theorem:2_1}
	Пусть $G$ — двусвязный граф без одиночных множеств.
	Тогда либо $G$ трёхсвязен, либо $G$ — это простой цикл
\end{thm}

\begin{proof}
	Пусть $G$ не трёхсвязен, тогда для каждого множества $S = \{a, b \} \in \R_2(G)$ и каждой части $A \in \Part(S)$ мы докажем, что $G(A)$ это простой $ab$-путь.
	Сразу заметим, что между $a$ и $b$ нет рёбер, ведь по лемме \ref{lemma:2_3} $G(A)$ имеет точку сочленения, отделяющую $a$ от $b$.
	А значит из утверждения индукции сразу последует что весь граф $G$ - простой цикл.

	Индукция по $|A|$.
	База при $|\Int(A)| = 1$ понятна, ведь внутренняя вершина $A$ соединена с $a$ и $b$.

	Переход: пусть $H = G(A)$, опять же по лемме \ref{lemma:2_3} $H$ имеет точку сочленения $x$, отделяющую $a$ от $b$.
	Пусть $U_a$ и $U_b$ компоненты связности графа $H - x$, содержащие $a$ и $b$ соответственно.
	Заметим, что других компонент в $H - x$ нет, ведь т.к. $x$ отделяет $a$ от $b$, то третья компонента $H - x$ была бы и в графе $G - x$, а он двусвязный.

	Пусть $U_a' = U_a \setminus \{a\} \neq \emptyset$.
	Тогда $R_a = \{a, x \}$ отделяет $U'_a$ от остальных вершин в графе $G$.
	Потому что $x$ отделяет от $U_b'$, а во все остальные(не из $A$) можно попасть только пройдя по $a$ либо по $b$, но $x$ также отделяет $U_a'$ и от $b$.

	Тогда, по предположению индукции граф $G(U_a' \cup R_a) = G(U_a \cup \{x\})$ — простой путь $ax$.

	Аналогично $G(U_b \cup \{x\})$ простой путь $bx$.

	А следовательно и $G(A)$ простой путь $ab$.
\end{proof}

\begin{thm}[Теорема 2.2] \label{theorem:2_2}
	Пусть $G$ - двусвязный граф.
	Тогда выполняются следующие утверждения:

	\begin{enumerate}
		\item Каждая часть $\Part(G)$ ~--- 3-блок или цикл
		\item Множество $R = \{a, b\}$ не одиночное множество из $\R_2(G)$, если и только если $a$ и $b$ - не соседние в циклическом порядке вершины некоторой части-цикла.
	\end{enumerate}

\end{thm}

\begin{proof}
	\begin{enumerate}
		\item Пусть $A \in \S$.
			Из Леммы \ref{lemma:1_2} следует что $G'(A)$ двусвязный.

			Предположим, что $S \in \R_2(G'(A))$.
			По Лемме \ref{lemma:1_2} тогда $S \in \R_2(G)$.
			Множество $S$ не может быть одиночным в $G$, т.к. разделяет часть $A \in \Part(G)$.
			Тогда по Лемме \ref{lemma:2_2}, $S$ - не одиночное разделяющее множество в  $G'(A)$.

			Следовательно, в  $G'(A)$ нет одиночных множеств.
			Тогда по теореме \ref{theorem:2_1} $G'(A)$ - цикл(тогда $A$-цикл) или трёхсвязный граф(тогда $A$ - 3-блок).

		\item $\Longleftarrow$. Пусть  $A = \{ a_1, a_2, \ldots, a_k \}$, причем вершины указаны в циклическом порядке, $R = \{a_1, a_m\}$, где  $2 < m < k$.

			Тогда  $R \in \R_2(G'(A))$ и делит граф  $G'(A)$ ровно на две части:

			\begin{align*}
				U_1 &= \{a_1, a_2, \ldots, a_m\} && U_2 = \{a_m, a_{m+1}, \ldots, a_k, a_1\}
			\end{align*}

			По Лемме \ref{lemma:1_2} имеем $R \in \R_2(G)$.

			И множество $R$ не одиночное т.к. можно взять пару вершин в $U_1, U_2$ отличных от $a, b$.

			$\Longrightarrow$. Пусть $R$ не одиночное множество, но оно независимо со всеми одиночными множествами графа  $G$, тогда $R$ лежит в одной из частей $A \in \Part(G)$.

			По Лемме \ref{lemma:2_2} тогда $R \in \R_2(G'(A))$.

			Из первого пункта теоремы  следует что $A$ не трёхсвязно, а значит $A$ - цикл длины хотя бы $4$.

			Следовательно  $R$ состоит из двух не соседних вершин этого цикла(иначе $R$ было бы одиночным).

	\end{enumerate}
\end{proof}


\begin{crly}[Следствие 2.1] \label{corollary:2_1}
	Пусть $G$ - двусвязный граф, $R \in \R_2(G)$. Тогда $R$ не содержит внутренних вершин частей-блоков и частей-треугольников(т.е. частей-циклов длины 3) графа $G$.
\end{crly}
\begin{proof}
	Пусть $B \in \Part(G)$ - 3-блок или треугольник.
	Если  $R$ - одиночное множество и $x \in R \cap B$, то $x \in \Bound(B)$(ведь мы разбиваем граф $G$ множеством $\D(G)$).
	
	Если  $R$ - не одиночное, то $R \cap \Int(B) = \emptyset$ по Теореме \ref{theorem:2_2}.
\end{proof}

\begin{crly}[Следствие 2.2] \label{corollary:2_2}
	Если часть $A \in \Part(G)$ - цикл, то все вершины из  $\Int(A)$ имеют степень 2 в графе $G$.
\end{crly}
\begin{proof}
	Если $x \in \Int(A)$, то рёбра графа $G$ выходят из $x$ только к вершинам части  $A$.
	А таких ребра всего два.
\end{proof}

\subsection{Дерево $B(G)$ и планарность}

\begin{df}[Подразбиение]
	Граф $H'$ называется \textbf{подразбиением} графа $H$, если  $H'$ может быть получен из  $H$ заменой некоторых рёбер на простые пути.
	Все добавляемые вершины различны и имеют степень 2. 

	Вершины графа $H'$, являющиеся вершинами графа  $H$(т.е. не являющиеся внутренними вершинами добавленных путей), называются \textbf{главными}.
\end{df}

\begin{prop}
	Через $G \supset H$ будем обозначать, что граф  $G$ содержит в качестве подграфа подразбиение графа  $H$.
\end{prop}

\begin{lm}[Лемма 2.4] \label{lemma:2_4}
	Пусть $G$ - двусвязный граф,  $A \in \Part(G)$.
	Тогда  $G \supset G'(A)$.
\end{lm}
\begin{proof}
	Пусть $ab \in E(G'(A)) \setminus E(G)$.
	Тогда  $a, b \in A$ и  $\{a, b\} \in \D(G)$. 

	Пусть $U_{a, b} \in \Part(\{a, b\})$ - часть, не содержащая $A$.

	Тогда существует  $ab$-путь $S_{a, b}$ по вершинам части  $U_{a, b}$ в графе $G$(потому что часть $S_{a, b}$ связна в $G$).
	Заменим ребро  $ab$ на этот путь  $S_{a, b}$ (т.е. мы как бы вместо одного прямого ребра $ab$ сделали длинной ребро-путь по вершинам не из $A$).

	Надо теперь показать что если там сделать для всех пар $ab \in E(G'(A)) \setminus E(G)$, то добавленные пути не будут пересекаться.

	В результате нескольких таких замен мы получим подграф $H$ графа  $G$.
	Пусть $ab$ и  $xy$ два разных замененных ребра(возможно они имеют общий конец).
	Тогда части  $U_{a, b}$ и  $U_{x, y}$ разделены частью  $A$ в  $\BT(G)$, поэтому не имею общей внутренней вершины.

	Следовательно, никакие два добавленных пути не имеют общей внутренней вершины, а значит, граф $H$ является подразбиением  $G'(A)$.
	
\begin{figure}[ht]
    \centering
	\incfig[0.5]{lemma_2_4}
	\caption{Пояснение к Лемме \ref{lemma:2_4}}
    \label{fig:lemma_2_4}
\end{figure}

\end{proof}


\begin{remrk}[Теорема Понтрягина — Куратовского] \label{theorem:pontyagin_kuratowski}
	Граф планарен тогда и только тогда, когда он не содержит подразбиений полного графа с пятью вершинами ($K_5$) и полного двудольного графа с тремя вершинами в каждой доле $K_{3,3}$.
\end{remrk}

\begin{thm}[Теорема 2.3, S. MacLane, 1937] \label{theorem:2_3}
	Пусть $G$ - двусвязный граф, а  $G' = G^{\D(G)}$.

	Тогда граф  $G$ планарен тогда и только тогда, когда для любого 3-блока  $B \in \Part(G)$, граф  $G'(B)$ планарен.
\end{thm}

\begin{proof}
	$\Longleftarrow$. Т.к. цикл планарен, то граф $G'(B)$ планарен просто для любой части  $B \in \Part(G)$.

	Пусть весь граф $G$ не планарен. Тогда по Теореме \ref{theorem:pontyagin_kuratowski} граф  $G$ имеет подграф  $H$ - подразбиение  $K_5$ или  $K_{3,3}$.

	Пусть $M$ - множество главных вершин  $H$.

	Поскольку и $K_5$ и  $K_{3,3}$ трёхсвязны, то любое двухвершинное разделяющее множество графа  $H$ не разделяет $M$.
	Поэтому существует часть  $B \in \Part(G)$, что  $B \supset M$.

	Предположим, что существует вершина $x \in V(H)$, что  $x \not \in B$, т.е. мы создали вершину $x$ при конструировании  $H$).
	А значит вершина $x$ лежит на пути  $S_{a, b}$ для каких то главных вершин $a, b \in B$.

	Пусть $x \in A \in \Part(G)$.
	Т.к. $\BT(G)$ - дерево, то существует смежное с  $B$ в  $BT(G)$(т.е. входящее в границу $B$) одиночное множество $R = \{y, y'\}$, отделяющие $A$ от  $B$.

	Тогда если мы пойдем по пути  $S_{a, b}$ от  $x$ в обе стороны, мы попадем в вершины множества  $R$.
	Тогда можно заменить участок пути между  $y, y'$ на ребро $yy'$ графа  $G'$.
	После нескольких таких операций вершины вне части  $B$ закончатся и мы получим граф  $H'$ - подразбиение  $K_5$ или  $K_{3,3}$ вершины которого лежат в  $B$.

	А т.к.  $H'$ - подграф  $G'(B)$, то значит  $G'(B)$ не планарен. Противоречие.

	 $\Longrightarrow$. Пусть  $G'(B)$ не планарен.
	 По Лемме \ref{lemma:2_4} существует подграф  $H$ графа  $G$, являющийся подразбиением  $G'(B)$.
	 Значит и  $H$ не планарен, а значит и  $G$.


\begin{figure}[ht]
    \centering
	\incfig[0.5]{theorem_2_3}
	\caption{Пояснение к первому пункту доказательства Теоремы \ref{theorem:2_3}}
    \label{fig:theorem_2_3}
\end{figure}

\end{proof}

\subsection{Критические двусвязные графы}

\begin{df}[Критический двусвязный граф]
	Назовем двусвязный граф  $G$ \textbf{критическим}, если он  теряет двусвязность при удалении любой вершины
\end{df}

\begin{crly}[Следствие 2.3] \label{corollary:2_3_1}
	Двусвязный граф $G$ является критическим  $\iff$ все его части-блоки и части-треугольники имеют пустую внутренность.
\end{crly}

\begin{proof}
	По Теореме \ref{theorem:2_2} вершины, не входящие в множества из $\R_2(G)$(т.е. удаление которых не нарушает двусвязность графа $G$) - это как раз внутренние вершины 3-блоков и треугольников графа $G$.
	Ведь при удалении вершины из цикла длины  $4$ мы точно сделаем его односвязным. 
\end{proof}

\begin{crly}[Следствие 2.3] \label{corollary:2_3_2}
	Пусть $A \in \Part(S)$ - крайняя часть критического двусвязного графа $G$, смежная в $\BT(G)$ с одиночным множеством  $S$.
	Тогда  $A$ - цикл длины хотя бы 4 и все вершины  $A$, кроме двух вершин множества  $S$, имеют в графе  $G$ степень 2.
\end{crly}

\begin{proof}
	Т.к. $A$ - крайняя часть графа  $G$, то  $A$ - цикл длины хотя бы 4(ведь треугольник и 3-блок не распадутся при удалении $S$), а  $S$ состоит из двух соседних вершин этого цикла(Теорема \ref{theorem:2_2}).
	Остальные (хотя бы две) вершины $A$ - внутренние и по Следствию \ref{corollary:2_2} имеют степень 2 в графе  $G$.

\begin{figure}[ht]
    \centering
	\includegraphics[width=0.2\columnwidth]{figures/corollary_2_3_2.png}
    \caption{Рисунок к Следствию \ref{corollary:2_3_2}}
    \label{fig:corollary_2_3_2}
\end{figure}

\end{proof}

\begin{crly}[Следствие 2.4, L. Nebesky, 1977] \label{corollary:2_4}
	Критический двусвязный граф $G : V(G) \geqslant 4$ имеет хотя бы 4 вершины степени ровно 2.
\end{crly}

\begin{proof}
	Если граф $G$ имеет хотя бы одно одиночное множество, то у него не менее двух крайних частей(одна часть это остаток цикла, а другая это просто какая то часть из оставшегося, содержащая $S$) и тогда по Следствию \ref{corollary:2_3_2} следует что у каждой такой части хотя бы две вершины степени 2.

	А если одиночных множеств в графе $G$ нет. То по теореме \ref{theorem:2_1} этот граф либо трёхсвязен, либо простой цикл.
	Критический двусвязный граф, очевидно, не может быть трёхсвязным.
	А значит граф  $G$ это простой цикл на хотя бы 4х вершинах.
\end{proof}

\subsection{Описание критических двусвязных графов с 4 вершинами степени 2}

Если $|V(G)| = 4$, то это цикл на четырех вершинах.

Если  $|V(G)| > 4$, то это не цикл длины 5(ведь тогда было бы 5 вершин степени 2), а значит  $G$ содержит одиночные множества \todo{что?}.

Дерево $\BT(G)$ должно иметь ровно две висячие вершины и они соответствуют циклам длины 4.
Следовательно, все не крайние части и все одиночные множества имеют степень 2 в $\BT(G)$.
Значит, каждое одиночное множество делит граф ровно на две части (для не одиночных множеств это всегда так).

Крайние части $G$ содержат ровно 4 внутренних вершины степени 2, следовательно, в некрайних частях вершин степени 2 нет.

Пусть $A \in \Part(G)$ - не крайняя часть.
Т.к. $d_{\BT(G)(A)} = 2$, то граница  $A$ состоит ровно из двух одиночных множеств, то есть, имеет 3 или 4 вершины.

\todo[inline]{ничего не понял в параграфах выше и не дописал}

\subsection{Удаление нескольких вершин с сохранением двусвязности}

Если из связного графа удалить любую внутреннюю вершину, то он останется связным.

Более того, если удалить из связного графа множество, состоящее из нескольких внутренних вершин блоков и содержащее не более чем по одной вершине каждого блока, то связность сохранится.
Обобщим утверждение для двусвязного графа.

Следствие \ref{corollary:2_1} говорит нам, что вершина двусвязного графа, удаление которой не нарушает двусвязности - это внутренняя вершина части-блока или части треугольника.

\begin{remrk}[Теорема Менгера] \label{theorem:menger}
	Наименьшее число вершин, разделяющих две не смежные вершины $s$ и  $t$, равно наибольшему числу не пересекающихся простых $(s - t)$ цепей.
\end{remrk}

\begin{thm}[Теорема 2.4] \label{theorem:2_4}
	Пусть $G$ - двусвязный граф, а $W$ - множество, состоящее из внутренних вершин не пустых 3-блоков графа  $G$ и содержащее не более чем по одной вершине из каждого 3-блока.
	Тогда граф  $G - W$ двусвязен.
\end{thm}

\begin{proof}
	От противного, пусть $W$ - минимальное по включение множество, вершины которого принадлежат внутренностям разных не пустых 3-блоков, а граф  $G^* = G - W$ не двусвязен.
	Понятно, что  $|W| \geqslant 2$(ведь удаления одной вершины 3-блока недостаточно для нарушения двусвязности).

	Т.к. вершины  $W$ принадлежат внутренностям разных  $3$-блоков, то существует одиночное множество  $S = \{a, b\} \in \D(G)$, разделяющее  $W$(одиночное, потому что мы разбиваем семейством $\D(G)$).
	И тогда $S \cap W = \emptyset$(ведь все вершины из $W$ внутренние).
 
	Части $\Part(S)$ можно разбить на две группы так, чтобы в каждой группе была часть, содержащая вершину из  $W$.
	Пусть  $U_1$ и  $U_2$ - объединения вершин этих групп и 
	\[
		U^* = V(G) \setminus W = V(G^*), \qquad U_1^* = U_1 \setminus W, \qquad U^*_2 = U_2 \setminus W
	\]

	Т.к. каждый 3-блок графа $G$ содержит хотя бы 4 вершины(иначе это цикл) и не более чем одна из них лежит в  $W$, то:

	\[
		|U_1^*| \geqslant 3, \qquad |U_2^*| \geqslant 3
	\] 

	Положим:

	\[
		G_1^* = G(U_1^*), \qquad G_2^* = G(U_2^*), \qquad G_1 = G_1^* + ab, \qquad G_2 = G_2^* + ab
	\] 

	Пусть $x \in U_2 \cap W$.
	По минимальности $W$ получаем, что граф  $G_x = G - (W \setminus \{x\})$ двусвязен.

	Множество $S$ отделяет  $U_1^*$ от  $U_2^* \cup \{x \}$ в двусвязном графе $G_x$.
	По Лемме \ref{lemma:1_2}(для набора из одного множества $S$) граф $G_1$ двусвязен.
	Аналогично,  $G_2$ двусвязен.

	От любой вершины  $y \in U^*$ в графе  $G^*$ существует  $ya$-путь  $P_a$ и  $yb$-путь  $P_b$ не имеющие общих вершин, кроме  $y$(по теореме Менгера \ref{theorem:menger} для двусвязного графа  $G_1$). 

	Нам достаточно доказать, что \textit{для любой вершины $v \in U^*$ в графе  $G^* - v$ все вершины  $U^* \setminus \{v\}$ связаны}(т.е. что выкидывание любой вершин оставляет граф связным, значит он он двусвязен).

	Рассмотрим любую вершину $u \in U^* \setminus S$.
	В графе  $G^*$ существует два не пересекающихся пути от  $u$ до вершин множества  $S$(т.е до $a$ и  $b$).
	Один из этих путей есть и в  $G^* - v$.

	Остается доказать, что при  $v \not \in S$ вершины  $a$ и  $b$ множества  $S$ связаны в графе  $G^* - v$(из этого следует, что из всех вершин $U^*$ есть путь то $a$ или $b$ и при этом есть  $ab$-путь).
	Без ограничений общности, $v \in U_1^*$.
	Тогда существует $ab$-путь $P$ в графе $G^* - v$, проходящий по вершинам из $U_2^*$, значит, $a$ и $b$ связаны в  $G^* - v$.

	Значит $G^*$ двусвязен, противоречие предположению.
	Следовательно, граф $G - W$ двусвязен для любого множества $W$, удовлетворяющего условию.
\end{proof}


\todo[inline]{Дописать лекцию 2}

