
\section{Структура двусвязного графа}

\subsection{Дерево разбиения двусвязного графа}

\todo[inline]{Дописать начало лекции 2}

\begin{df}[Дерево разбиения $\BT(G)$]
	\textbf{Дерево разбиения} $\BT(G)$ двусвязного графа  $G$ - это дерево  $T(G, \D(G))$, см Определение \ref{definition:tree_of_partition}.

	Будем обозначать $\Part(G) = \Part(\D(G))$.
	Частями графа  $G$ будем называть элементы  $\Part(G)$.
\end{df}

\begin{df}[Крайняя часть]
	Часть $A \in \Part(\D(G))$ назовем \textbf{крайней}, если она соответствует висячей вершине дерева разбиения  $\BT(G)$.
\end{df}

\begin{df}[G']
	Для двусвязного графа $G$ обозначим через $G'$ граф, полученный из $G$ добавлением всех отсутствующих в $E(G)$ ребер вида $ab$, где $\{a, b\} \in \D(G)$.

	Т.е. $G' = G^{\D(G)}$
\end{df}

\begin{df}[Цикл и 3-блок]
	Назовем часть $A \in \Part(G)$ \textbf{циклом}, если граф $G'(A)$ - простой цикл и \textbf{3-блоком}, если граф $G'(A)$ трёхсвязен.
	Если часть $A$ - цикл, то мы будем называть $|A|$ \textbf{длиной} цикла.
\end{df}

\begin{lm}[Лемма 2.2] \label{lemma:2_2}
	Пусть $S \in \R_2(G)$ - не одиночное множество, $S \subset A \in \Part(G)$. 
	Тогда $S \in \R_2(G'(A))$, причем это множество - не одиночное и в  $G'(A)$.
\end{lm}

\begin{proof}
	Пусть $S' \in \R_2(G)$ - зависимое с  $S$ множество.

	По Лемме \ref{lemma:1_1} получаем что $S, S' \in \R_2(G')$ и эти множества зависимы в графе $G'$.

	По Лемме \ref{lemma:1_2} $G'(A)$ двусвязный, а значит нельзя разделить две вершины множества  $S \subset A$ в графе  $G'$, удалив менее двух вершин из части  $A$.

	Следовательно,  $S' \subset A$(ведь он разделяет $S$ и при этом $|S'| = 2$).

	Тогда $S$ и  $S'$ разделяют друг друга и в графе $G'(A)$.

	Следовательно,  $S, S' \in \R_2(G'(A))$, причем эти множества зависимы.

\end{proof}

\begin{lm}[Лемма 2.3] \label{lemma:2_3}
	Пусть $S = \{ a, b \} \in \R_2(G)$ не одиночное множество.
	Тогда $|\Part(S)| = 2$ и для каждой части $A \in \Part(S)$ граф $G(A)$ имеет точку сочленения, отделяющую $a$ от $b$.
\end{lm}

\begin{proof}
	Т.к. $S$ не одиночное, то возьмем зависимое с ним $S' \in \R_2(G)$.

	Множество $S'$ разделяет $S$, а значит не существует $ab$ пути по вершинам части $A$ в графе $G$ который не пересекается с $S'$(потому что тогда бы $S'$ не отделяло $a$ от $b$).
	Но если $S'$ не пересекает $\Int(A)$, то такой путь есть(ведь $A$ связно).

	А значит всего частей в $\Part(S)$ ровно две(ведь их внутренности не пересекаются).

	Более того, если $\{x\} = S' \cap \Int(A)$, то $x$ — отделяет $a$ от $b$ в $G(A)$ по сказанному выше.
	
\end{proof}

\begin{thm}[Теорема 2.1] \label{theorem:2_1}
	Пусть $G$ — двусвязный граф без одиночных множеств.
	Тогда либо $G$ трёхсвязен, либо $G$ — это простой цикл
\end{thm}

\begin{proof}
	Пусть $G$ не трёхсвязен, тогда для каждого множества $S = \{a, b \} \in \R_2(G)$ и каждой части $A \in \Part(S)$ мы докажем, что $G(A)$ это простой $ab$-путь.
	Сразу заметим, что между $a$ и $b$ нет рёбер, ведь по лемме \ref{lemma:2_3} $G(A)$ имеет точку сочленения, отделяющую $a$ от $b$.
	А значит из утверждения индукции сразу последует что весь граф $G$ - простой цикл.

	Индукция по $|A|$.
	База при $|\Int(A)| = 1$ понятна, ведь внутренняя вершина $A$ соединена с $a$ и $b$.

	Переход: пусть $H = G(A)$, опять же по лемме \ref{lemma:2_3} $H$ имеет точку сочленения $x$, отделяющую $a$ от $b$.
	Пусть $U_a$ и $U_b$ компоненты связности графа $H - x$, содержащие $a$ и $b$ соответственно.
	Заметим, что других компонент в $H - x$ нет, ведь т.к. $x$ отделяет $a$ от $b$, то третья компонента $H - x$ была бы и в графе $G - x$, а он двусвязный.

	Пусть $U_a' = U_a \setminus \{a\} \neq \emptyset$.
	Тогда $R_a = \{a, x \}$ отделяет $U'_a$ от остальных вершин в графе $G$.
	Потому что $x$ отделяет от $U_b'$, а во все остальные(не из $A$) можно попасть только пройдя по $a$ либо по $b$, но $x$ также отделяет $U_a'$ и от $b$.

	Тогда, по предположению индукции граф $G(U_a' \cup R_a) = G(U_a \cup \{x\})$ — простой путь $ax$.

	Аналогично $G(U_b \cup \{x\})$ простой путь $bx$.

	А следовательно и $G(A)$ простой путь $ab$.
\end{proof}

\begin{thm}[Теорема 2.2] \label{theorem:2_2}
	Пусть $G$ - двусвязный граф.
	Тогда выполняются следующие утверждения:

	\begin{enumerate}
		\item Каждая часть $\Part(G)$ ~--- 3-блок или цикл
		\item Множество $R = \{a, b\}$ не одиночное множество из $\R_2(G)$, если и только если $a$ и $b$ - не соседние в циклическом порядке вершины некоторой части-цикла.
	\end{enumerate}

\end{thm}

\begin{proof}
	\begin{enumerate}
		\item Пусть $A \in \S$.
			Из Леммы \ref{lemma:1_2} следует что $G'(A)$ двусвязный.

			Предположим, что $S \in \R_2(G'(A))$.
			По Лемме \ref{lemma:1_2} тогда $S \in \R_2(G)$.
			Множество $S$ не может быть одиночным в $G$, т.к. разделяет часть $A \in \Part(G)$.
			Тогда по Лемме \ref{lemma:2_2}, $S$ - не одиночное разделяющее множество в  $G'(A)$.

			Следовательно, в  $G'(A)$ нет одиночных множеств.
			Тогда по теореме \ref{theorem:2_1} $G'(A)$ - цикл(тогда $A$-цикл) или трёхсвязный граф(тогда $A$ - 3-блок).

		\item $\Longleftarrow$. Пусть  $A = \{ a_1, a_2, \ldots, a_k \}$, причем вершины указаны в циклическом порядке, $R = \{a_1, a_m\}$, где  $2 < m < k$.

			Тогда  $R \in \R_2(G'(A))$ и делит граф  $G'(A)$ ровно на две части:

			\begin{align*}
				U_1 &= \{a_1, a_2, \ldots, a_m\} && U_2 = \{a_m, a_{m+1}, \ldots, a_k, a_1\}
			\end{align*}

			По Лемме \ref{lemma:1_2} имеем $R \in \R_2(G)$.

			И множество $R$ не одиночное т.к. можно взять пару вершин в $U_1, U_2$ отличных от $a, b$.

			$\Longrightarrow$. Пусть $R$ не одиночное множество, но оно независимо со всеми одиночными множествами графа  $G$, тогда $R$ лежит в одной из частей $A \in \Part(G)$.

			По Лемме \ref{lemma:2_2} тогда $R \in \R_2(G'(A))$.

			Из первого пункта теоремы  следует что $A$ не трёхсвязно, а значит $A$ - цикл длины хотя бы $4$.

			Следовательно  $R$ состоит из двух не соседних вершин этого цикла(иначе $R$ было бы одиночным).

	\end{enumerate}
\end{proof}


\begin{crly}[Следствие 2.1] \label{corollary:2_1}
	Пусть $G$ - двусвязный граф, $R \in \R_2(G)$. Тогда $R$ не содержит внутренних вершин частей-блоков и частей-треугольников(т.е. частей-циклов длины 3) графа $G$.
\end{crly}
\begin{proof}
	Пусть $B \in \Part(G)$ - 3-блок или треугольник.
	Если  $R$ - одиночное множество и $x \in R \cap B$, то $x \in \Bound(B)$.
\todo[inline]{Лемму о том что одиночное множество не пересекает ничью внутренность, возможно было на лекциях}
	
	Если  $R$ - не одиночное, то $R \cap \Int(B) = \emptyset$ по Теореме \ref{theorem:2_2}.
\end{proof}

\begin{crly}[Следствие 2.2] \label{corollary:2_2}
	Если часть $A \in \Part(G)$ - цикл, то все вершины из  $\Int(A)$ имеют степень 2 в графе $G$.
\end{crly}
\begin{proof}
	Если $x \in \Int(A)$, то рёбра графа $G$ выходят из $x$ только к вершинам части  $A$.
	А таких ребра всего два.
\end{proof}

\subsection{Дерево $B(G)$ и планарность}

\begin{df}[Подразбиение]
	Граф $H'$ называется \textbf{подразбиением} графа $H$, если  $H'$ может быть получен из  $H$ заменой некоторых рёбер на простые пути.
	Все добавляемые вершины различны и имеют степень 2. 

	Вершины графа $H'$, являющиеся вершинами графа  $H$(т.е. не являющиеся внутренними вершинами добавленных путей), называются \textbf{главными}.
\end{df}

\begin{prop}
	Через $G \supset H$ будем обозначать, что граф  $G$ содержит в качестве подграфа подразбиение графа  $H$.
\end{prop}

\begin{lm}[Лемма 2.4] \label{lemma:2_4}
	Пусть $G$ - двусвязный граф,  $A \in \Part(G)$.
	Тогда  $G \supset G'(A)$.
\end{lm}
\begin{proof}
	Пусть $ab \in E(G'(A)) \setminus E(G)$.
	Тогда  $a, b \in A$ и  $\{a, b\} \in \D(G)$. 

	Пусть $U_{a, b} \in \Part(\{a, b\})$ - часть, не содержащая $A$.

	Тогда существует  $ab$-путь $S_{a, b}$ по вершинам части  $U_{a, b}$ в графе $G$(потому что часть $S_{a, b}$ связна в $G$).
	Заменим ребро  $ab$ на этот путь  $S_{a, b}$ (т.е. мы как бы вместо одного прямого ребра $ab$ сделали длинной ребро-путь по вершинам не из $A$).

	Надо теперь показать что если там сделать для всех пар $ab \in E(G'(A)) \setminus E(G)$, то добавленные пути не будут пересекаться.

	В результате нескольких таких замен мы получим подграф $H$ графа  $G$.
	Пусть $ab$ и  $xy$ два разных замененных ребра(возможно они имеют общий конец).
	Тогда части  $U_{a, b}$ и  $U_{x, y}$ разделены частью  $A$ в  $\BT(G)$, поэтому не имею общей внутренней вершины.

	Следовательно, никакие два добавленных пути не имеют общей внутренней вершины, а значит, граф $H$ является подразбиением  $G'(A)$.
	
\begin{figure}[ht]
    \centering
	\incfig[0.5]{lemma_2_4}
	\caption{Пояснение к Лемме \ref{lemma:2_4}}
    \label{fig:lemma_2_4}
\end{figure}

\end{proof}

\todo[inline]{Дописать лекцию 2}

