\newpage

\section{Стягиваемые и связные множества}

\subsection{Теорема Дьори-Ловаса р разбиении $k$-связного графа на $k$ связных множеств заданных размеров.}


\begin{figure}[ht]
    \centering
	\incfig[0.4]{lemma_5_2}
    \caption{Рисунок к Лемме 5.2}
    \label{fig:lemma_5_2}
\end{figure}

\begin{proof}[\normalfont\textsc{Доказательство Утверждения 1}]
	"Рассмотрим все $r$-оптимальные конфигурации с хвостом ранга $r$" -> понятно, что если есть хвост ранга $s$, то значит есть и конфигурация с рангом  $s$, вот из таких и выберем минимум назвав его  $r$.
\end{proof}

\begin{proof}[\normalfont\textsc{Доказательство Леммы 1}]
	последний слайд с док-вом леммы. у нас тут уже случай что нет хвостов, т.к. если они есть то по утв 1 мы бы доказали уже лемму.

	т.к. нет хвостов, то S не смежно с A(вернее можно быть смежно только по последней вершине каскада, а это $W$)(S не смежно с $V_1$, т.к. если смежно то просто добавим)

	А $B$ может быть смежно только с  $W$ по утв 2
\end{proof}


\begin{proof}[\normalfont\textsc{Доказательство Леммы 3}]
	Граф $H$ не трёхсвязен, т.к.  $W$ ~-- максимально
\end{proof}

\subsection{Максимальные стягиваемые множества в техсвязном графе (определение, лемма о удалении).}


по сути мы когда удаляем и хотим чтобы была связность $k - 1$ то это примерно тоже самое что стянуть и оставить связность  $k$. справа налево следствие точно есть

\subsection{Существование стягиваемого множества из трёх вершин в трёхсвязном графе.}


\todo[inline]{почему внутренность каждой части $\Part(H_W)$ содержит хотя бы две вершины? может если это не так то мы называем часть циклом? хотя вроле обычно длина 4 нужна}

\begin{proof}[\normalfont\textsc{Доказательство Теоремы 2}]
	"Тогда $p \geq 3 + |\Int(A')| \geq 5$" -> это из того что $p \geq |\Int(A)| + |\Int(A')|$, а т.к. мы выбирали $W$ минимальным по окрестности, то значит что у любого $W$ окрестность это хотя бы 5 вершин.

	"В таком случае, $ya_1 \in E(G)$ или $ya_3 \in E(G)$" -> т.к. у $y$ должно быть как минимум два соседа среди  $a_1, a_2, a_3$.

	В начале случая 2.2, $H_W - L$ ~-- двусвязен по лемме 3
\end{proof}


\subsection{Теорема о стягиваемых множествах больших размеров в трёхсвязном графе.}


\begin{proof}[\normalfont\textsc{Доказательство Леммы 4}]
	В первом случае, когда пишем неравенство на размер $|W'|$, то $|W \setminus \{x, y\}| \leqslant m- 1$, т.к.  $x, y$ могут совпасть.
\end{proof}


