
\section{Стягивание ребер в $k$-связном графе}

\section{Связные и стягиваемые множества вершин в $k$-связном графе}

\todo[inline]{заполнить 4-9 лекции}

Выяснили, что вершины любого $k$-связного графа можно разбить на  $k$ связных множеств заданных размеров.

\begin{df}[Стягиваемое множество]
	Множество $W$ в $k$-связном графе $G$ при $k \geqslant 2$ называется \textbf{стягиваемым}, если граф $G - W$ является $k - 1$ связным.
\end{df}

Логично задаться вопросом, всегда ли в графе найдется стягиваемое множество, хотя бы в случае когда размер стягиваемого множества мал по сравнению с общим количеством вершин в графе.
Из работ Мадера следует, что при $k \geqslant 4$ ответ отрицательный: для любого  $m$ существуют сколь угодно большие $k$-связные графы, не имеющие стягиваемого множества из $m$ вершин. 

\begin{df}[Максимальное стягиваемое множество]
	Пусть $W$ - стягиваемое множество вершин трёхсвязного графа $G$. Назовем $W$ \textbf{максимальным стягиваемым множеством}, если не существует такой вершины $x \in V(G) \setminus W$, что множество  $W \cup \{x\}$ ~-- стягиваемое.
\end{df}

\begin{remrk}
	Если стягиваемое множество $W$ ~-- максимальное, то для любой смежной с $W$ вершины $x \in V(G) \setminus W$ граф  $G - W - x$ не двусвязен.
\end{remrk}

\begin{lm}[Лемма 5.3] \label{lemma:5_3}
	Пусть $G$ - трёхсвязный граф, $W \subset V(G)$ - максимальное стягиваемое множество, причем $H = G - W$ не является простым циклом. Тогда выполняются следующие утверждения:

	\begin{enumerate}
		\item Множество $W$ смежно со всеми внутренними вершинами частей-циклов графа $H$
		\item Все крайние части  $\Part(H)$ - циклы длины не менее 4. Крайних частей в $\Part(H)$ не менее двух. Граница любой крайней части является одиночным множеством графа $H$
		\item Пусть $A$ - крайняя часть $\Part(H)$, тогда граф $H - \Int(A)$ двусвязен
	\end{enumerate}

\end{lm}

\begin{proof}
	\begin{enumerate}
		\item Внутренние вершины частей-циклов графа $H$ имеют степень 2 в $H$, поэтому в трёхсвязном графе $G$ они должны быть смежны с $W$.
	\end{enumerate}
\end{proof}
