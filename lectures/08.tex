\newpage

\section{Восстановление графа}

\subsection{Теорема Келли о восстановлении дерева.}

\subsection{Реконструкция крайних блоков в графе связности 1 (Леммы 1-3).}

\begin{proof}[\normalfont\textsc{Доказательство Леммы 2}]
	в $\implies$, "Так как $\delta(G) \geq 2$" -> верно, т.к. рассмотрели случай $m = 1$ в б.и.
\end{proof}

\subsection{Теорема Бонди о реконструкции графа связности 1 и минимальной степени хотя бы 3.}

\subsection{Свойства частей разбиения двусвязного графа 2-разделяющим множеством (Леммы 4 и 5, Следствие 1).}

\subsection{Удаление вершины из минимального крайнего 3-блока: разбиение полученного графа его границей (Лемма 7).}

\subsection{Лемма о свойствах графа $G(D)^*$ (Лемма 9).}

\subsection{Вес ребра и дуги. Лемма о части, соответствующей дуге цикла.}

\todo[inline]{в самом низу слайда определения, что две несоседние вершины части-цикла не образуют разделяюшего множества. Кажется наоборот ведь, две соседние не образуют а две не соседние образуют(но не одиночное), не стоит этого писать на экзе. А, видимо одиночными будут те две которые крепят цикл к основному графу.}

\subsection{Восстановление крайних 3-блоков.}

\subsection{Свойства графов из $\mathcal D(b_1)$. Утверждения 1 - 4.}

\subsection{Случай, когда $T_1$ неодиночное множетсво в $G - x$. Утверждения 5 и 6.}

\subsection{Случай, когда невозможно определить правильную дугу и $B_1 - x$. Утверждения 7 и 8.}

\subsection{Окончание доказательства теоремы о восстановлении двух графов по колоде.}
