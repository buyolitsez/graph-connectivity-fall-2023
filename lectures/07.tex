\newpage

\section{Компоненты зависимости}

\subsection{Компоненты зависимости. Часть, содержащая компоненту зависимости (Леммы 1 и 2, определение, Следствие 1).}

\begin{proof}[\normalfont\textsc{Доказательство Леммы 1}]
	Когда мы получаем что $S_1$ и $S_2$ независимы, то у нас противоречие. Тут я бы лучше сказал что если найдутся такие $S_1$ и $S_2$, то найдутся и такие же соединенные ребром в нашем графе $\Dep(\S)$, а вот они значит явно зависимы.
\end{proof}

\subsection{Гипердерево компонент зависимости.}

Пояснение к предисловию определения:
Как мы знаем, каждая компонента из $\Comp(\S)$ содержится полностью в одной части  $\Part(\T)$, тогда у нас классы это как раз части  $\Part(\T)$.

Тем самым в $\Struct(\S)$ мы считаем две компоненты соседними, если никакая третья их не разделяет.
Тогда гиперребра нашего графа это все максимальные по включению множества попарно соседних вершин. Собственно вершины это сами компоненты.

\subsection{Лемма и следствие о границе части (Следствие 2 и Лемма 3).}
\subsection{Часть, соответствующая гиперребру (определение и Лемма 4).}

В определении части гиперребра $R$ на самом деле эта вещь не всегда часть.

\begin{proof}[\normalfont\textsc{Доказательство Леммы 4}]
	$A_{i \supset l} \neq A_i$ т.к. $A_i$ содержит все из $R \setminus \{i\}$, а  $A_{i \supset l}$ содержит $\S_l \not \in R$. 

	"Поскольку для всех $i \in \{1, \dots, n\} \; \; A_i \supset \cup_{j = 1}^{n}\Bound(A_j)$" -> верно, т.к. $A_i$ содержит все компоненты гиперребра, кроме  $i$.

	"Случай 2: Тогда из скзанного выше следует, что $\Bound(A_1) = \Bound(A_2) = \dots = \Bound(A_n) = A_R \in \S$" -> мы выше показали что либо $A_R \in \Part(\S)$, либо $A_R$ ~-- подмножество одного из набора  $\S$, у нас как видно второй случай, значит  $|A_R| \leq k \implies |A_R| = k$ т.к. он содержит все границы то значит они все и равны. Ну а граница размера  $k$ это разделяющее множество.

	конец доква: "граф зависимости $\Dep(\S_2 \cup \S_3)$ связен, что невозможно" невозможно т.к. они разные компоненты зависимости
\end{proof}

\subsection{Части разбиения и гипердерево компонент зависимости (Теорема 2).}

\begin{proof}[\normalfont\textsc{Доказательство Теоремы 2}]
	Крайней вершиной гипердерева называем такую $\T$, что она не разделяет никакие две другие.
\end{proof}

\subsection{Теорема о границе части (Теорема 3).}
