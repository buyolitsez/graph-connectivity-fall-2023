\newpage

\setcounter{section}{-1}
\section{Разделяющие множества и части разбиения}

\subsection{Разбиение графа набором разделяющих множеств: часть, ее граница и внутренность.}

\begin{df*}[Часть разбиения]
	Пусть $\S$ — набор из нескольких разделяющих множеств графа $G$ (которые могут содержать как вершины, так и ребра графа $G$), тогда $A \subset V(G)$ это \textbf{часть $G$-разбиения}, если никакие две вершины из $A$ нельзя разделить никаким множеством из $\S$, но любая другая вершина графа $G$ отделена от множества $A$ хотя бы одним из множеств набора $\S$. 
\end{df*}

\begin{prop*}
	Множество всех частей разбиения графа $G$ набором $\S$ обозначим как $\Part(\S)$ или $\Part(G; \S)$.
\end{prop*}

\begin{df*}[Внутренняя и граничная вершины]
	Вершину $v$ части $A \in \Part(\S)$ назовем \textbf{внутренней}, если она не входит ни в одно множество из $\S$ и назовем \textbf{границей} иначе.

	Множество всех внутренних вершин $A$ обозначим $\Int(A)$, а границ $\Bound(A)$.

	Внутренняя вершина части $A \in \Part(\S)$ может быть концом ребра, входящего в множество  $S \in \S$.
\end{df*}

\begin{prop*}
	Через $\R(G)$ обозначим семейство всех разделяющих множеств графа $G$(вершинных и реберных).

	Через $\R_k(G)$ обозначим семейство всех $k$-вершинных разделяющих множеств графа $G$.
\end{prop*}

\begin{customthm}{0.1} \label{theorem:0_1}
	Пусть $\S \subset \R(G), A \in \Part(\S)$, а граф  $G'$ получен из  $G$ удалением всех входящих в множества набора  $\S$ рёбер.
	Тогда выполняются следующие утверждения:
	 \begin{enumerate}
		 \item Вершина $x \in \Int(A)$ не смежна ни с одной из вершин множества  $V(G) \setminus A$ графа  $G'$
		\item Если  $\Int(A) \neq \varnothing$, то  $\Bound(A)$ отделяет  $\Int(A)$ от $V(G) \setminus A$ в графе  $G'$
	\end{enumerate}
\end{customthm}
\begin{proof}
	\begin{enumerate}
		\item Пусть $x \in \Int(A)$ смежна с  $y \in V(G) \setminus A$.
			Но существует множество  $S \in \S$, отделяющие  $y$ от  $\Int(A)$ в  $G$.

			Т.к. $x$ лежит в $\Int(A)$, то  $x \not \in S$.
			И $y$ не лежит в  $S$, ведь  $S$ отделяет  $x$ от  $y$.
			Получаем что для того чтобы разделить $x$ и  $y$ надо удалить ребро  $xy \implies xy \in S$.

		\item Следует из первого пункта. Ведь если удалить $\Bound(A)$, то тогда не будет ребер изнутри наружу.
	\end{enumerate}
\end{proof}

\begin{customcrly}{1}\label{corollary:0_1}
	Пусть $\S$ набор разделяющих множеств в графе  $G$, содержащий только вершины(т.е. в $\S$ нет ребер).
	И $A \in \Part(\S), \Int(A) \neq \varnothing$.
	Тогда  $\Bound(A)$ отделяет $\Int(A)$ от $V(G) \setminus A$ в графе  $G$.
\end{customcrly}

\begin{remrk} \label{remark:0_inside_connected_with_bound}
	Пусть $G$ - $k$-связный граф, $S \in \R_k(G), A \in \Part(S)$ и $x \in S$.

	Тогда существует $y \in \Int(A)$, смежная с $x$.

	Ведь иначе бы $S \setminus \{x\}$ отделяло бы  $\Int(A)$ от $G - A$ и тогда граф не был $k$-связным(т.е. при удалении $S$ граф разваливается на части, а если вернуть вершину $x$, то он опять становится связным).
\end{remrk}

\begin{customthm}{0.2}\label{theorem:0_2}
	Пусть $G$ ~-- $k$-связный граф, $\S \subset \R_k(G)$.
	Тогда для любой $A \in \Part(\S)$ граница $\Bound(A)$ есть множество всех вершин части  $A$, смежных хотя бы с одной вершиной из $V(G) \setminus A$. 
\end{customthm}
\begin{proof}
	Пусть $x \in \Bound(A)$, т.е. существует $S \in \S \colon x \in S$.

	Множество вершин $S$ не разделяет $A$, следовательно $A$ пересекает внутренность не более чем одной части $\Part(S)$.
	Т.е. существует часть  $B \in \Part(S) \colon \Int(B) \cap A = \varnothing$.
	А значит существует вершина  $y \in \Int(B)$ смежная с  $x$, по Замечанию \ref{remark:0_inside_connected_with_bound}.

	И в обратную сторону: по Следствию \ref{corollary:0_1} ни одна из вершин $\Int(A)$ не может быть смежна с вершиной из  $V(G) \setminus A$.
\end{proof}

\begin{crly*}
	Тогда для двух разбиений $\S, \T \subset \R_k(G)$ границы  $A$  внутри этих разбиений совпадают.
	Ведь Теорема \ref{theorem:0_2} задает границу без привязки к разбиению.
\end{crly*}

\subsection{Представление части разбиения графа набором множеств в виде пересечения частей разбиения поднаборами.}

\begin{customthm}{0.3} \label{theorem:0_3}
	Пусть $\S_1, \dots, \S_n \subset \R(G)$, а $\S = \cup_{i = 1}^{n} \S_i$. Рассмотрим все множества вершин вида

	\begin{align}
		A = \bigcap_{i = 1}^{n} A_i, \quad \text{где } A_i \in \Part(\S_i) \label{eq:0_thm_3}
	\end{align}
	Тогда выполняются следующие утверждения:

	\begin{enumerate}
		\item \label{enum:0_thm_3_1} Любая часть $A \in \Part(\S)$ представляется в виде \eqref{eq:0_thm_3}
		\item \label{enum:0_thm_3_2}  $A \in \Part(\S) \iff$  $A$ ~-- максимальное по включению множество вершин графа  $G$, представимое в виде \eqref{eq:0_thm_3}
		\item \label{enum:0_thm_3_3} Если множество вершин представимо в виде \eqref{eq:0_thm_3} и $A \not \in \Part(\S)$, то  $A$ является подмножеством одного из множеств набора  $\S$
	\end{enumerate}

\end{customthm}

\begin{proof}
	\begin{enumerate}
		\item Пусть $A \in \Part(\S)$, тогда для любого  $i \in [n]$ ни одно множество из набора  $\S_i$ не разделяет  $A$, следовательно, существует часть $A_i$ для любого $i \colon A_i \in \Part(\S_i)$ содержащая  $A$. Давайте пересечем эти части, каждая из которых содержит $A$.
			
			Обозначим  $A' = \cap_{i = 1}^{n} A_i \implies A \subset A'$.

			Никакое множество набора $\S$ не разделяет  $A'$, следовательно, существует часть  $B \in \Part(\S)$, что  $A' \subset B$, таким образом  $A \subset A' \subset B$, откуда следует, что  $A = A' = B$, ведь $A \in \Part(\S)$. 
		\item $\Longleftarrow$. Пусть множество $A \subset V(G)$, представимое в виде \eqref{eq:0_thm_3} ~-- максимальное по включению среди множеств такого вида. Тогда  $A$ невозможно разделить никаким множеством набора  $\S$. И если  $A \not \in \Part(S)$, то найдется часть  $B \in \Part(\S) \colon B \supsetneq A$, тогда  $B$ представима в виде \eqref{eq:0_thm_3} по пункту \eqref{enum:0_thm_3_1}  $\implies $ !?

			$\Longrightarrow$. Пусть  $A \in \Part(\S)$, рассмотрим представление  $A$ в виде  \eqref{eq:0_thm_3}. Пусть $A$ не максимальное по включению среди множеств такого виду. Тогда  $A \subset B$, где  $B$ представимо в виде \eqref{eq:0_thm_3}. Тогда  $B$ не разделяется множеством из набора  $\S$, а следовательно  $A \not \in \Part(\S)$.

		\item Пусть  $A = \cap_{i = 1}^{n} A_i$, где $A_i \in \Part(\S_i)$ и  $A \not \in \Part(\S)$.
			Тогда существует часть  $B \in \Part(S) \colon B \supsetneq A$.

			Рассмотрим представление  $B$ из пункта \eqref{enum:0_thm_3_2}:  $B = \cap_{i = 1}^{n} B_i$, $B_i \in \Part(\S_i)$. Т.к.  $B \neq A$, то найдется $j \colon A_j \neq B_j \implies A \supset A_j \supset B_j$, а их пересечение является подмножеством одного из  $\S_j$.
	\end{enumerate}
	
\end{proof}

\begin{customlm}{0.1}
	Пусть $\S, \T \subset \R(G)$, а часть  $A \in \Part(\S)$ такова, что ни одно из множеств набора  $\T$ ее не разделяет.

	Тогда  $A \in \Part(\S \cup \T)$.
\end{customlm}

\begin{proof}
	Т.к. ни одно из множеств набора $\S \cup \T$ не разделяет  $A$, то найдется часть  $B \in \Part(\S \cup \T) \colon A \subset B$.

	Тогда существует содержащая  $B$ часть  $A' \in \Part(\S)$, а тогда  $A \subset B \subset A' \implies A = B$.
\end{proof}

\subsection{Зависимые и независимые k-разделяющие множества. Свойства}

\begin{df*}[Независимые множества]
	Назовем различные множества $S, T \in \R_k(G)$ \textbf{независимыми}, если $S$ не разделяет $T$ и $T$ не разделяет $S$.
	В противном случае назовем их \textbf{зависимыми}.
\end{df*}

\begin{customlm}{0.2} \label{lemma:0_2}
	Пусть $S, T \in \R_k(G)$ и часть  $A \in \Part(S)$ такова, что  $T \cap \Int(A) = \varnothing \implies T$ не разделяет часть  $A$ и следовательно,  $T$ не разделяет множество  $S$.
\end{customlm}

\begin{proof}
	Понятно, что граф $G(\Int(A))$ связен, а любая вершина  $x \in S \setminus T$ смежна хотя бы с одной из вершин множества  $\Int(A)$.

	Следовательно, граф  $G(\Int(A) \cup (S \setminus T))$ связен, откуда очевидно следует, что  $T$ не разделяет  $A$ и  поэтому $T$ не разделяет  $S$(ведь $S \subset A$).
\end{proof}

\begin{customlm}{0.3} \label{lemma:0_3}
	Пусть $S, T \in \R_k(G)$ таковы, что множество  $S$ не разделяет множество $T$. Тогда множество $T$ не разделяет множество $S$(т.е. они независимы).
\end{customlm}

\begin{proof}
	Т.к. $S$ не разделяет $T$, то множество $T$ может пересекать внутренность не более чем одной из частей $\Part(S)$.

	Тогда существует часть  $A \in \Part(S) \colon \Int(A) \cap T = \varnothing$.
	По Лемме \ref{lemma:0_2} $T$ не разделяет~$S$.
\end{proof}

\begin{df*}[Одиночное множество]
	Назовем \textbf{одиночными} множествами из $\R_k(G)$, не зависимые ни с какими другими множествами из $\R_k(G)$.
	Через $\D(G)$ обозначим семейство одиночных множеств. 
\end{df*}

\begin{customlm}{0.4} \label{lemma:0_4}
	Пусть множества $S, T \in \R_k(G)$ независимы, а часть  $A \in \Part(S)$ содержит  $T$.
	Тогда в  $\Part(T)$ есть часть, содержащая все отличные от  $A$ части из $\Part(S)$, а все остальные части $\Part(T)$ являются подмножествами  $A$.
\end{customlm}

\begin{figure}[ht]
    \centering
	\incfig[0.4]{lemma_0_4}
	\caption{Расположение частей в Лемме \ref{lemma:0_4}.}
    \label{fig:lemma_0_4}
\end{figure}

\begin{proof}
	Множество $T$ не пересекает внутренностей отличных от  $A$ частей  $\Part(S)$(ведь содержится в $A$), тогда по Лемме \ref{lemma:0_2} множество $T$ не разделяет никакой отличной от  $A$ части $\Part(S)$.

	По аналогии $S$ не может разделять никакие части  $\Part(T)$ кроме той, в которой сама содержится. А значит что $A$ содержит все остальные части  $\Part(T)$.

\end{proof}

\begin{customlm}{0.5} \label{lemma:0_5}
	Пусть множества $S, T \in \R_k(G)$ зависимы,  $\Part(S) = \{A_1, \ldots, A_m\}, \, \Part(T) = \{B_1, \ldots, B_n\}, \; P = T \cap S, \; T_i = T \cap \Int(A_i), \; S_j = S \cap \Int(B_j), \; G_{i, j} = A_i \cap B_j$.

	Тогда выполняются следующие утверждения:

	\begin{enumerate}
		\item Все множества $T_1, \ldots, T_m, S_1, \ldots, S_n$ непусты
		\item $\Part(\{S, T\}) = \{G_{i, j}\}$, причем  $\Bound(G_{i, j}) = P \cup T_i \cup S_j$
	\end{enumerate}

\end{customlm}

\begin{proof}
	\begin{enumerate}
		\item По Лемме \ref{lemma:0_2} и зависимости $S$ и  $T$ получаем, что  $T_i \neq \varnothing, S_j \neq \varnothing$, ведь иначе  $S$ и  $T$ были бы независимы.
		\item По Теореме \ref{theorem:0_3} части  $\Part(\{S, T\})$ это максимальные по включению среди множеств вида $G_{i, j}$, но из пункта 1 следует что  $G_{\alpha, \beta} \not \subset G_{\gamma, \delta}$ при  $(\alpha, \beta) \neq (\gamma, \delta)$, поэтому и части $\{S, T\}$ это $G_{i, j}$.
 
			Утверждение  $\Bound(G_{i, j}) = P \cup T_i \cup S_j$ очевидно.
	\end{enumerate}
\end{proof}

\subsection{Разрез, границы разреза. Разбиения графа набором разрезов. Граничный разрез части.}

\begin{df*}[Разрезы]
	Пусть $G$ -  $k$-связный граф.
	
	Будем называть \textbf{разрезом}  $k$-элементное разделяющее множество из вершин и рёбер графа  $G$, содержащее хотя бы одно ребро.
	Множество всех разрезов графа  $G$ обозначим через  $\T(G)$.

	Для разреза  $T \in \T(G)$ обозначим через  $V(T)$ множество всех входящих в  $T$ вершин, а через  $W(T)$ множество, состоящее из всех вершин, входящих в разрез  $T$ и всех вершин, инцидентных рёбрам разреза  $T$.
\end{df*}

Отметим, что никакая вершина из $V(T)$ не инцидентна ребру из  $T$, ведь тогда бы граф был  $<k$-связным.

И для любого разреза $T \in T(G)$, граф  $G - T$ имеет ровно две компоненты связности, ведь добавление ребра из  $T$ делает граф опять связным. 
И тогда каждая часть содержит по одному концу для любого ребра из $T$.

\begin{df*}[Части разбиения]
	Пусть $T \in \T(G)$,  $U_1, U_2$ - компоненты связности графа $G - T$.
	Назовем множества  $A_i = U_i \cup V(T)$ - \textbf{частями разбиения} графа  $G$ разрезом  $T$.

	Обозначаем,  $\Part(T) = \{A_1, A_2\}$.
\end{df*}

\begin{df*}[Границы разреза]
	\textbf{Границами разреза}  $T$ назовем множества вершин  $R(A_1) = A_1 \cap W(T)$ и $R(A_2) = A_2 \cap W(T)$.
\end{df*}


Пусть $T \in \T(G), \ \Part(T) = \{A_1, A_2\}$.

Заметим пару простых вещей:

\begin{itemize}
	\item $A_1 \cup A_2 = V(G), A_1 \cap A_2 = V(T)$
	\item Границы разреза $T$ содержат по  $k$ элементов, ведь каждая из границ разреза $T$ содержит  $V(T)$ и по одному концу всех входящих в  $T$ рёбер.
	\item Если множество $A' = A_1 \setminus W(T)$ не пусто, то $R(A_1)$ отделяет $A'$ от  $V(G) \setminus A_1$, а каждая вершина из $x \in R(A_1)$ смежна хотя бы с одной вершиной из $A'$.
		Таким образом,  $R(A_1) \in \R_k(G)$, в этом случае.
\end{itemize}


\begin{df*}[Квазичасть]
	Пусть $\S \subset \T(G)$.
	Назовем \textbf{квазичастями} разбиения графа $G$ множеством разрезов  $\S$ множества вида:

	\[
		A = \bigcap_{S \in \S} A_S, \quad \text{где } A_S \in \Part(S)
	\] 
 
Будем обозначать множество всех квазичастей разбиения графа $G$ множеством разрезов $\S$ через $\QPart(\S)$.

\end{df*}

\begin{df*}[Граница квазичасти]
	\textbf{Границей квазичасти} $A$ будет множество  $\Bound(A)$ всех вершин этой квазичасти, являющихся вершинами разрезов из  $\S$.
\end{df*}

\begin{df*}[Внутренность квазичасти]
	\textbf{Внутренностью квазичасти} $A$ назовем множество  $\Int(A) = A \setminus \Bound(A)$.
\end{df*}

\begin{df*}[Граничный разрез]
	Определим \textbf{граничный разрез} $\Cut(A)$ квазичасти  $A$: он состоит из множества вершин  $\Bound(A)$ и всех рёбер, входящих в разрезы множества  $\S$ и инцидентных вершинам из  $\Int(A)$.
\end{df*}

Опять заметим некоторые факты про квазичасти:

\begin{itemize}
	\item По Теореме \ref{theorem:0_3}:

		- части $\Part(\S)$ - это максимальные по включения квазичасти из  $\QPart(G; \S)$

		- если  $B \in \QPart(\S), B \not \in \Part(\S)$, то $B \subset V(S)$ для некоторого разреза  $S \in \S$. Ведь $\Int(B) = \varnothing$.

		- если  $\S = \S_1 \cup \S_2$, то части  $\Part(\S)$ это максимальные по включению множества вершин, представимые в виде  $A = A_1 \cap A_2$, где $A_i \in \Part(\S_i)$
	
	\item Если две различные квазичасти $A_1, A_2 \in \QPart(\S)$ имеют не пустое пересечение, то $A_1 \cap A_2 \subset V(S)$ для некоторого разреза $S \in S$
	
	\item Граничный разрез части не обязательно является разрезом.
		Для этого необходимо, чтобы  $\Cut(A)$ содержал ровно  $k$ элементов и среди них было хотя бы одно ребро.
\end{itemize}

\begin{customlm}{0.6} \label{lemma:0_6}
	Пусть $\S \subset \T(G), A \in \Part(\S)$, причем  $\Int(A) \neq \varnothing$.
	Обозначим через  $\overline A$ объединение всех отличных от  $A$ частей  $\Part(\S)$.
	Тогда выполняются следующие утверждения:

	\begin{enumerate}
		\item $\Cut(A)$ отделяет  $A$ от  $\overline A$
		\item  $\overline A \setminus \Bound(A) = \overline A \setminus A \supset V(G) \setminus A$
		\item Если  $|\Cut(A)| = k$ и  $\Cut(A)$ содержит хотя бы одно ребро, то  $\Cut(A)$ - разрез,  $\Part(\Cut(A)) = \{A, A'\}$, где $A' \subset \overline A$.
			Если  $A' \setminus \overline A \neq \varnothing$, то это множество состоит из некоторых вершин  $\Bound(A)$, инцидентных рёбрам, входящим в разрезы из  $\S$.
	\end{enumerate}

\end{customlm}

\begin{proof}
	\begin{enumerate}
		\item Любое ребро  $e \in E(G)$, выходящее из  $\Int(A)$ в  $\overline A \setminus \Bound(A)$ соединяет две вершины, разделенные хотя бы одним из разрезов множества  $\S$, а значит и само принадлежит хотя бы одному из разрезов  $\S$. Но тогда  $e \in \Cut(A)$. И тогда $\Cut(A)$ отделяет  $A$ от  $\overline A$.

		\item Т.к.  $\overline A \cap \Int(A) = \varnothing$, то  $\overline A \setminus \Bound(A) = \overline A \setminus A$.
			А т.к.  $A \cup \overline A = V(G)$, то  $V(G) \setminus A = \overline A \setminus A$.
		\item Из первых двух пунктов следует, что $\Cut(A)$ отделяет друг от друга не пустые множества  $\Int(A)$ и  $V(G) \setminus A$. Поэтому из условия следует, что $\Cut(A)$ ~-- разрез.

			Понятно, что  $A \in \Part(\Cut(A))$, пристальнее взглянем на  $A'$ теперь.

			По пункту 1 имеем  $\overline A \subset A'$, пусть  $x \in A' \setminus \overline A$, тогда  $x \in \Bound(A)$(ведь $x$ не может лежать внутри  $A$ и оно не лежит в  $\overline A$).

			Вершина $x$ смежна с  $y \in V(G) \setminus A$(иначе $\Bound(A) \setminus\{x\}$) отделяет $\Int(A)$ от  $V(G) \setminus A$, что противоречит $k$-связности графа).
			Если ребро $xy$ не входит в разрезы  $\S$, то существует часть  $B \in \Part(\S)$, содержащая $x$ и $y$. Т.к. $B \neq A$ и  $x \in B \subset \overline A$, то получаем противоречие.
			А значит $xy$ входит в какой то разрез  $\S$ и  $A'$ имеет вид указанный в условии.
	\end{enumerate}
\end{proof}

\subsection{Независимые разрезы. Свойства.}

\begin{df*}[Независимые и зависимые разрезы]
	Разрезы $S, T \in \T(G)$, такие что $\Part(S) = \{A_1, A_2\}, \ \Part(T) = \{B_1, B_2\}$, что $A_1 \supset B_2$ и $B_1 \supset A_2$, называются \textbf{независимыми}.
	Иначе, назовем $S$ и  $T$ \textbf{зависимыми}.
\end{df*}

\begin{figure}[ht]
    \centering
	\incfig[0.5]{df_0_independent_cuts}
	\caption{Независимые разрезы.}
    \label{fig:df_0_independent_cuts}
\end{figure}

\begin{customlm}{0.7} \label{lemma:0_7}
	Пусть разрезы $S, R, T \in \T(G)$ таковы, что  $S$ и  $R$ независимы, а также $T$ и  $R$ независимы.
	Пусть  $\Part(S) = \{A_1, A_2\}, \, \Part(T) = \{B_1, B_2\}, \ \Part(R) = \{D_1, D_2\}$, причем $D_1 \supset A_1$ и $D_2 \supset B_2$.
	Тогда разрезы $S$ и  $T$ независимы.
\end{customlm}

\begin{proof}
	Из независимости $S$ и  $R$ получаем что  $A_2 \supset D_2 \supset B_2$, а из независимости $T$ и  $R$ следует, что  $B_1 \supset D_1 \supset A_1$, таким образом $S$ и  $T$ независимы.
\end{proof}

\begin{customlm}{0.8} \label{lemma:0_8}
	Пусть разрезы $S, T \in \T(G)$ независимы,  $\Part(S) = \{A_1, A_2\}, \Part(T) = \{B_1, B_2\}$, причем $A_1 \supset B_2$ и $B_1 \supset A_2$. Тогда выполнено следующее:

	\begin{enumerate}
		\item $A_1 \supset R(B_2), \; A_2 \not \supset R(B_2)$
		\item Если $S$ и  $T$ не имеют общих рёбер, то  $A_1 \supset W(T)$ и $A_2 \not \supset R(B_1)$
	\end{enumerate}

\end{customlm}

\begin{figure}[ht]
	\centering
	\incfig[0.5]{lemma_0_8}
	\caption{Рисунок к Лемме \ref{lemma:0_8}. Закрашенная область соответствует $R(B_2)$.} 
	\label{fig:lemma_0_8}
\end{figure}


\begin{proof}
	\begin{enumerate}
		\item Очевидно, что $A_1 \supset B_2 \supset R(B_2)$.

		Предположим, что $A_2 \supset R(B_2)$, тогда 

		\begin{align*}
			k - 1 \geqslant |V(S)| = |A_1 \cap A_2| \geqslant |R(B_2)| = k \implies \text{!?}
		\end{align*}

		\item Пусть $b_1b_2 \in T, b_i \in \Int(B_i)$, знаем что $b_2 \in B_2 \subset A_1$.

			Если $b_2 \in S$, то $b_2 \in A_2 \subset B_1$, что неверно.
			Значит $b_2 \not \in S$, поскольку $b_1b2 \not \in S$(по условию), то вершины $b_1$ и $b_2$ не разделены разрезом $S$, т.е. $b_1 \in A_1$ и следовательно $A_1 \supset W(T)$.

			Теперь предположим, что $A_2 \supset R(B_1)$, тогда 

			\begin{align*}
				k - 1 \geqslant |V(S)| = |A_1 \cap A_2| \geqslant |R(B_1)| = k \implies \text{!?}
			\end{align*}

	\end{enumerate}
\end{proof}
