
\setcounter{section}{-1}
\section{Разделяющие множества и части разбиения}

\begin{df}[Часть разбиения]
	Пусть $\S$ — набор из нескольких разделяющих множеств графа $G$ (которые могут содержать как вершины, так и ребра графа $G$), тогда $A \subset V(G)$ это \textbf{часть $G$-разбиения}, если никакие две вершины из $A$ нельзя разделить никаким множеством из $\S$, но любая другая вершина графа $G$ отделена от множества $A$ хотя бы одним из множеств набора $\S$. 

	Т.е. $\forall (a, b) \in A, \forall S \in \S$ существует путь $ab$ в $G \setminus S$ и $\forall a \not \in A, \exists b \in A, \exists S \in \S$ что не существует пути $ab$ в $G \setminus S$.
\end{df}

\begin{prop}
	Множество всех частей разбиения графа $G$ набором $\S$ обозначим как $\Part(\S)$ или $\Part(G; \S)$.
\end{prop}

\begin{df}[Внутренняя и граничная вершины]
	Вершину $v$ части $A \in \Part(\S)$ назовем \textbf{внутренней}, если она не входит ни в одно множество из $\S$ и назовем \textbf{границей} иначе.

	Множество всех внутренних вершин $A$ обозначим $\Int(A)$, а границ $\Bound(A)$.
\end{df}

\begin{prop}
	Через $\R(G)$ обозначим семейство всех разделяющих множеств графа $G$(вершинных и реберных).

	Через $\R_k(G)$ обозначим семейство всех $k$-вершинных разделяющих множеств графа $G$.
\end{prop}

\begin{thm}[Теорема 0.1] \label{theorem:0_1}
	Пусть $\S \subset \R(G), A \in \Part(\S)$, а граф  $G'$ получен из  $G$ удалением всех входящих в множества набора  $\S$ рёбер.
	Тогда выполняются следующие утверждения:
	 \begin{enumerate}
		 \item Вершина $x \in \Int(A)$ не смежна ни с одной из вершин множества  $V(G) \setminus A$ графа  $G'$
		\item Если  $\Int(A) \neq \emptyset$, то  $\Bound(A)$ отделяет  $\Int(A)$ от $V(G) \setminus A$ в графе  $G'$
	\end{enumerate}
\end{thm}
\begin{proof}
	\begin{enumerate}
		\item Пусть $x \in \Int(A)$ смежна с  $y \in V(G) \setminus A$.
			Но существует множество  $S \in \S$, отделяющие  $y$ от  $\Int(A)$ в  $G$.

			Т.к. $x$ лежит в $\Int(A)$, то  $x \not \in S$.
			И $y$ не лежит в  $S$, ведь  $S$ отделяет  $x$ от  $y$.
			Получаем что для того чтобы разделить $x$ и  $y$ надо удалить ребро  $xy \implies xy \in S$.

		\item Следует из первого пункта. Ведь если удалить $\Bound(A)$, то тогда не будет ребер изнутри наружу.
	\end{enumerate}
\end{proof}

\begin{crly}[Следствие 1]\label{corollary:0_1}
	Пусть $\S$ набор разделяющих множеств в графе  $G$, содержащий только вершины(т.е. в $\S$ нет ребер).
	И $A \in \Part(\S), \Int(A) \neq \emptyset$.
	Тогда  $\Bound(A)$ отделяет $\Int(A)$ от $V(G) \setminus A$ в графе  $G$.
\end{crly}

\begin{remrk}
	Пусть $G$ - $k$-связный граф, $S \in \R_k(G), A \in \Part(S)$ и $x \in S$.

	Тогда существует $y \in \Int(A)$, смежная с $x$.

	Ведь иначе бы $S \setminus \{x\}$ отделяло бы  $\Int(A)$ от $G - A$ и тогда граф не был $k$-связным(т.е. при удалении $S$ граф разваливается на части, а если вернуть вершину $x$, то он опять становится связным).
\end{remrk}

\begin{thm}[Теорема 0.2] \label{theorem:0_2}
	Пусть $G$ - $k$-связный граф, $\S \subset \R_k(G)$.
	Тогда для любой $A \in \Part(\S)$ граница $\Bound(A)$ есть множество всех вершин части  $A$, смежных хотя бы с одной вершиной из $V(G) \ A$. 
\end{thm}
\begin{proof}
	Пусть $x \in \Bound(A)$, т.е. существует $S \in \S \colon x \in S$.

	Множество вершин $S$ не разделяет $A$, следовательно $A$ пересекает внутренность не более чем одной части $\Part(S)$.
	Т.е. существует часть  $B \in \Part(S) \colon \Int(B) \cap A = \emptyset$.
	А значит существует вершина  $y \in \Int(B)$ смежная с  $x$.

	И в обратную сторону: по Следствию \ref{corollary:0_1} ни одна из вершин $\Int(A)$ не может быть смежна с вершиной из  $V(G) \setminus A$.
\end{proof}

\begin{crly}
	Тогда для двух разбиений $\S, \T \subset \R_k(G)$ границы  $A$  внутри этих разбиений совпадают.
	Ведь Теорема \ref{theorem:0_2} задает границу без привязки к разбиению.
\end{crly}

\subsection{Зависимые и независимые разделяющие множества}

\begin{df}[Независимые множества]
	Назовем различные множества $S, T \in \R_k(G)$ \textbf{независимыми}, если $S$ не разделяет $T$ и $T$ не разделяет $S$.
	В противном случае назовем \textbf{зависимыми}.
\end{df}

\begin{df}[Одиночное множество]
	Назовем \textbf{одиночными} множествами из $\R_k(G)$, не зависимые ни с какими другими множествами из $\R_k(G)$.
	Через $\D(G)$ обозначим семейство одиночных множеств. 
\end{df}

\subsection{Разбиение $k$-связного графа парой зависимых $k$-разделяющих множеств}

\subsection{Разрезы}

\subsection{Части разбиения графа множеством разрезов}

\subsection{Независимые разрезы}

\todo[inline]{Дописать лекцию 0}
