\newpage

\section{$k$-соединенные графы}

\subsection{Лемма о “плотном” подграфе.}

\begin{df*}
	Обозначим $\delta(G)$ ~-- минимальную степень вершину в графе  $G$.
\end{df*}

\subsection{Лемма о соединенном множестве: утверждения 1 и 2 (о соединенном разбиении).}
\subsection{Лемма о соединенном множестве: утверждение 3 (о общих соседях).}

\begin{proof}[\normalfont\textsc{Доказательство Утверждения 3.2}]
	во второй точке, мы заменяем $u \cdot v$ на пару вершин  $u, v$ если это ребро лежало в  $A'$ и ничего не делаем иначе.
\end{proof}

\subsection{Лемма о соединенном множестве: утверждение 4 о минимальной степени и конец доказательства леммы. Теорема о $k$-соединеннности графа.}


\begin{proof}[\normalfont\textsc{Доказательство Утверждения 4}]
	Наверное изолированные надо выкидывать до того как определять $d*$
\end{proof}

\begin{df*}
	Через $\kappa(G)$ обозначим максимальное число $t\colon$ граф  $G$ ~-- $t$ вершинно связный.
\end{df*}

\begin{proof}[\normalfont\textsc{Доказательство Теоремы 1}]
	$\kappa(G) \geq 2k$, т.к. по условию теоремы  $G$ ~-- $2k$-связный.
\end{proof}

